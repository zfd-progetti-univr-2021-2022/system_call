\subsection*{Come suddividere messaggi non divisibili per 4}

D\+O\+M\+A\+N\+DA\+:

Siccome l\textquotesingle{}ultima parte del messaggio e\textquotesingle{} l\textquotesingle{}unica che puo\textquotesingle{} essere piu\textquotesingle{} corta per specifica... Cosa bisogna fare in casi in cui non e\textquotesingle{} possibile garantire questo vincolo?

Esempio\+: 2 caratteri possono essere divisi in\+:
\begin{DoxyItemize}
\item caratteri per parte\+: 1 1 0 0
\item oppure\+: 2 0 0 0
\end{DoxyItemize}

Lo stesso problema si pone per 1, 2, 5, 6, 9, 10, ... caratteri.

Non posso garantire come ad esempio nel caso di 3 caratteri che sono l\textquotesingle{}ultimo numero sia inferiore\+:

Esempio di suddivisione di 3 caratteri\+: 1 1 1 0. L\textquotesingle{}ultimo, come per specifica, e\textquotesingle{} l\textquotesingle{}unico di dimensione inferiore

R\+I\+S\+P\+O\+S\+TA\+:

Va bene l\textquotesingle{}algoritmo attualmente utilizzato\+: il contenuto di un file viene diviso in parti uguali e intere tra i canali di comunicazione mentre il resto viene distribuito un pezzo per canale di comunicazione.

Esempi\+:

\tabulinesep=1mm
\begin{longtabu} spread 0pt [c]{*{5}{|X[-1]}|}
\hline
\rowcolor{\tableheadbgcolor}\textbf{ Numero caratteri}&\textbf{ Caratteri per F\+I\+FO 1}&\textbf{ Caratteri per F\+I\+FO 2}&\textbf{ Caratteri per msgqueue}&\textbf{ Caratteri per memoria condivisa  }\\\cline{1-5}
\endfirsthead
\hline
\endfoot
\hline
\rowcolor{\tableheadbgcolor}\textbf{ Numero caratteri}&\textbf{ Caratteri per F\+I\+FO 1}&\textbf{ Caratteri per F\+I\+FO 2}&\textbf{ Caratteri per msgqueue}&\textbf{ Caratteri per memoria condivisa  }\\\cline{1-5}
\endhead
0 &0 &0 &0 &0 \\\cline{1-5}
1 &1 &0 &0 &0 \\\cline{1-5}
2 &1 &1 &0 &0 \\\cline{1-5}
3 &1 &1 &1 &0 \\\cline{1-5}
4 &1 &1 &1 &1 \\\cline{1-5}
5 &2 &1 &1 &1 \\\cline{1-5}
6 &2 &2 &1 &1 \\\cline{1-5}
7 &2 &2 &2 &1 \\\cline{1-5}
8 &2 &2 &2 &2 \\\cline{1-5}
9 &3 &2 &2 &2 \\\cline{1-5}
10 &3 &3 &2 &2 \\\cline{1-5}
... &... &... &... &... \\\cline{1-5}
\end{longtabu}
\subsection*{dimensione massima percorsi file}

D\+O\+M\+A\+N\+DA\+:

I percorsi dei file hanno dimensione massima?

R\+I\+S\+P\+O\+S\+TA\+:

Non sappiamo la lunghezza massima dei percorsi dei file ma sappiamo che sono molto corti.

I test verranno eseguiti nella cartella {\ttfamily /home} oppure {\ttfamily /runner} e ci saranno al massimo 3 o 4 sottocartelle\+: probabilmente non si superera\textquotesingle{} il centinaio di caratteri. \begin{quote}
Attualmente vengono supportati percorsi di lunghezza massima di 255 caratteri \end{quote}


\subsection*{Serve la documentazione?}

D\+O\+M\+A\+N\+DA\+:

La specifica non richiede la documentazione. E\textquotesingle{} richiesta?

R\+I\+S\+P\+O\+S\+TA\+:

Non e\textquotesingle{} mai stata richiesta, quindi in teoria no.

\subsection*{Caratteri non A\+S\+C\+II}

D\+O\+M\+A\+N\+DA\+:

I caratteri nei file di testo in input ai client sono A\+S\+C\+II? Sono tutti da 1 byte? Bisogna gestire lettere accentate, ...?

R\+I\+S\+P\+O\+S\+TA\+:

I caratteri che verranno letti sono tutti da un byte e di conseguenza non c\textquotesingle{}e\textquotesingle{} da preoccuparsi di caratteri non A\+S\+C\+II. 