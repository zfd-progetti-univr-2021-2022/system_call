Le chiavi sono dei numeri interi che servono ad alcune system call per creare una I\+PC (se non esiste gia\textquotesingle{}) e restituire il suo identificatore che permette poi di utilizzarla per comunicare con altri processi. \begin{quote}
Le I\+PC sono meccanismi di comunicazione tra processi. Puo\textquotesingle{} essere la memoria condivisa, la coda F\+I\+FO, la P\+I\+PE, i segnali, i semafori, le code di messaggi. \end{quote}


Per fare un\textquotesingle{}analogia con i file di testo e\textquotesingle{} possibile vedere le chiavi come il percorso del file e l\textquotesingle{}identificatore come il file descriptor\+:
\begin{DoxyItemize}
\item I primi servono per creare i secondi
\item I secondi permettono di gestire il file/la I\+PC
\end{DoxyItemize}

I processi che conoscono la chiave possono comunicare fra di loro creando ed utilizzando le I\+PC.

Per creare le chiavi esistono piu\textquotesingle{} modi diversi.

\subsection*{Macro I\+P\+C\+\_\+\+P\+R\+I\+V\+A\+TE}

La macro {\ttfamily I\+P\+C\+\_\+\+P\+R\+I\+V\+A\+TE} permette di generare una chiave che sara\textquotesingle{} sicuramente univoca perche\textquotesingle{} il kernel ce lo garantira\textquotesingle{}.

Esempio di creazione di un semaforo utilizzando la macro\+: 
\begin{DoxyCode}
\textcolor{keywordtype}{id} = semget(IPC\_PRIVATE, 10, S\_IRUSR | S\_IWUSR);
\end{DoxyCode}


E\textquotesingle{} comoda quando un processo padre crea la I\+PC prima di creare i processi figli con cui comunichera\textquotesingle{} in seguito.

E\textquotesingle{} difficile utilizzarla quando i processi che devono comunicare non sono imparentati.

\subsection*{Funzione file to key}

La funzione {\ttfamily ftok()} (file to key) converte il percorso di un file ed un identificativo di progetto in una chiave.


\begin{DoxyCode}
\textcolor{preprocessor}{#include <sys/ipc.h>}
\textcolor{comment}{// Returns integer key on succcess, or -1 on error (check errno)}
key\_t ftok(\textcolor{keywordtype}{char} *pathname, \textcolor{keywordtype}{int} proj\_id);
\end{DoxyCode}
 \begin{quote}
Vengono usati solo gli ultimi 8 bit di proj\+\_\+id \end{quote}


Il file nel percorso pathname deve essere esistente e accessibile (bisogna avere i permessi giusti)


\begin{DoxyCode}
Esempio:
key\_t key = ftok(\textcolor{stringliteral}{"/mydir/myfile"}, \textcolor{charliteral}{'a'});
\textcolor{keywordflow}{if} (key == -1)
    errExit(\textcolor{stringliteral}{"ftok failed"});

\textcolor{keywordtype}{int} \textcolor{keywordtype}{id} = semget(key, 10, S\_IRUSR | S\_IWUSR);

\textcolor{keywordflow}{if} (\textcolor{keywordtype}{id} == -1)
    errExit(\textcolor{stringliteral}{"semget failed"});
\end{DoxyCode}


Questa opzione e\textquotesingle{} comoda per far comunicare processi non imparentati mantenendo un punto di riferimento fisso.

\subsection*{Inserimento manuale}

Per progetti semplici posso scegliere di fissare un numero per gestire le I\+PC.

Questa opzione e\textquotesingle{} comoda per far comunicare processi non imparentati in progetti piccoli. 