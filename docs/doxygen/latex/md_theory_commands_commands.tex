\subsection*{Comandi debugger G\+DB}

Per prima cosa bisogna compilare i file sorgente con le informazioni di debugging con la flag {\ttfamily -\/g}.

Esempio\+: 
\begin{DoxyCode}
# Usa -g per permettere il debugging
gcc -g <nome>.c -o <output>
\end{DoxyCode}


Comandi G\+DB\+:
\begin{DoxyItemize}
\item {\ttfamily break $<$nome funzione$>$}\+: imposta un breakpoint su una funzione
\item {\ttfamily break $<$riga codice$>$}\+: imposta un breakpoint su una certa riga del codice
\item {\ttfamily clear $<$riga codice$>$}\+: toglie il breakpoint su quella riga di codice
\item {\ttfamily watch $<$nome variabile$>$}\+: aggiunge un watchpoint che blocca il codice quando la variabile cambia valore
\item {\ttfamily run}\+: esegue il programma
\item {\ttfamily print $<$variabile$>$}\+: visualizza il valore della variabile $>$ E\textquotesingle{} possibile anche scrivere print var\+\_\+array\mbox{[}x\mbox{]} per visualizzare l\textquotesingle{}elemento di un array
\item {\ttfamily step}\+: esegue la riga di codice successiva seguendo il codice nelle funzioni
\item {\ttfamily next}\+: esegue la riga di codice successiva eseguendo le funzioni in un colpo solo
\item {\ttfamily continue}\+: esegue il codice finche\textquotesingle{} non viene raggiunto un breakpoint o watchpoint
\item {\ttfamily list}\+: visualizza la riga
\item {\ttfamily quit}\+: esce da gdb
\item {\ttfamily signal $<$segnale$>$}\+: invia al processo il segnale {\ttfamily $<$segnale$>$} $>$ Esempio\+: {\ttfamily signal S\+I\+G\+I\+NT}, utile per fare debugging del client del progetto
\end{DoxyItemize}

\subsection*{Estensioni dell\textquotesingle{}interfaccia G\+DB e interfacce grafiche}

\subsubsection*{Interfacce C\+LI}

\href{https://github.com/cyrus-and/gdb-dashboard}{\tt gdb-\/dashboard} estende l\textquotesingle{}interfaccia del terminale.



E\textquotesingle{} possibile scaricando il file {\ttfamily .gdbinit} nella cartella home\+: 
\begin{DoxyCode}
wget -P ~ https://git.io/.gdbinit
\end{DoxyCode}


Per aggiungere il syntax highlight bisogna installare Pygments (libreria Python)\+: 
\begin{DoxyCode}
pip install pygments
\end{DoxyCode}


\subsubsection*{Interfacce G\+UI}

\href{https://github.com/cs01/gdbgui}{\tt gdbgui} e\textquotesingle{} una interfaccia grafica basata su tecnologie web.



\subsection*{Ricerca memory leak}

Valgrind e\textquotesingle{} uno strumento che permette di analizzare la memoria usata dai processi per trovare memory leak oppure in generale accessi errati della memoria.

Testare il server\+: 
\begin{DoxyCode}
valgrind --leak-check=full \(\backslash\)
    --show-leak-kinds=all \(\backslash\)
    --track-origins=yes \(\backslash\)
    --verbose \(\backslash\)
    --log-file=valgrind-server-out.txt \(\backslash\)
    ./server
\end{DoxyCode}


Testare il client\+: 
\begin{DoxyCode}
valgrind --leak-check=full \(\backslash\)
    --show-leak-kinds=all \(\backslash\)
    --track-origins=yes \(\backslash\)
    --verbose \(\backslash\)
    --log-file=valgrind-client-out.txt \(\backslash\)
    ./client\_0 <percorso cartella file>
\end{DoxyCode}
 \begin{quote}
Sostituire {\ttfamily $<$percorso cartella file$>$} con l\textquotesingle{}input corretto \end{quote}


\subsection*{Visualizza file aperti}

Per visualizzare file aperti dai processi (tra cui le F\+I\+FO) e\textquotesingle{} possibile utilizzare il programma {\ttfamily lsof}.

Per installarlo su debian\+: 
\begin{DoxyCode}
sudo apt install lsof
\end{DoxyCode}


Per visualizzare tutti i file aperti eseguire il comando\+: 
\begin{DoxyCode}
lsof
\end{DoxyCode}


Per visualizzare i file aperti di un solo processo conoscendo il suo P\+ID\+: 
\begin{DoxyCode}
lsof -p <PID>
\end{DoxyCode}


\subsection*{Visualizza informazioni I\+PC}

Il comando {\ttfamily ipcs} permette di vedere una lista di\+:
\begin{DoxyItemize}
\item code dei messaggi
\item segmenti di memoria condivisa
\item set di semafori
\end{DoxyItemize}

Esempio di output\+: 
\begin{DoxyCode}
------ Message Queues --------
key     msqid  owner    perms  used-bytes  messages
0x1235  26     student  620    12          20

------ Shared Memory Segments --------
key     shmid  owner      perms  bytes  nattch  status
0x1234  0      professor  600    8192   2

------ Semaphore Arrays --------
key     semid  owner      perms  nsems
0x1111  102    professor  330    20
\end{DoxyCode}


\subsection*{Rimuovi I\+PC dal sistema\+:}

Il comando {\ttfamily ipcrm} permette di rimuovere le I\+PC.

Rimuovere una coda dei messaggi\+: 
\begin{DoxyCode}
# 0x1235 e' la chiave della coda
ipcrm -Q 0x1235

# 25 e' l'identificatore della coda
ipcrm -q 26
\end{DoxyCode}


Rimuovere un segmento di memoria condivisa\+: 
\begin{DoxyCode}
# 0x1234 e' la chiave della memoria condivisa
ipcrm -M 0x1234

# 0 e' l'identificatore della memoria condivisa
ipcrm -m 0
\end{DoxyCode}


Rimuovere un set di semafori\+: 
\begin{DoxyCode}
# 0x1111 e' la chiave del set di semafori
ipcrm -S 0x1111

# 102 e' l'identificatore del set di semafori
ipcrm -s 102
\end{DoxyCode}
 