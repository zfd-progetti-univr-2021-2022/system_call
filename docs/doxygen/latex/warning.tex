
\begin{DoxyRefList}
\item[\label{warning__warning000001}%
\Hypertarget{warning__warning000001}%
Globale \hyperlink{client_8h_a8c7084a254c7cd640d66e647795ff8f6}{operazioni\+\_\+client0} ()]Per ottimizzare l\textquotesingle{}uso dello H\+E\+AP nel client 0 si potrebbe prima cercare e contare quanti file sono presenti senza creare una lista concatenata e poi ricercare i file e man mano che si trovano file send\+\_\+me si puo\textquotesingle{} creare il processo figlio per inviare il file. Per fare questo B\+I\+S\+O\+G\+NA sapere se il numero di file puo\textquotesingle{} cambiare durante l\textquotesingle{}esecuzione di questa funzione\+: se trovo 3 file e dopo un file viene cancellato cosa succede? ~\newline
 N\+O\+TA\+: questo problema puo\textquotesingle{} esserci anche nella situazione attuale...

Il client 0 deve attendere i processi figlio? La specifica indica solo che bisogna attendere il messaggio di fine dal server... Attualmente prima si attendere il messaggio di fine e poi si aspetta che tutti i figlio terminino.

Il percorso passato al client deve essere assoluto o puo\textquotesingle{} essere relativo? Se si passa un percorso relativo chdir() fallira\textquotesingle{} alla seconda esecuzione. ~\newline
 S\+O\+L\+U\+Z\+I\+O\+NE\+: si potrebbe usare un altro chdir() a fine funzione per tornare al percorso di esecuzione iniziale anticipando il chdir() successivo.
\end{DoxyRefList}