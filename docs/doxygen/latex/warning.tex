
\begin{DoxyRefList}
\item[\label{warning__warning000001}%
\Hypertarget{warning__warning000001}%
File \hyperlink{client_8c}{client.c} ]La specifica non richiede la documentazione. E\textquotesingle{} richiesta?

I percorsi dei file hanno dimensione massima?

I caratteri nei file di testo in input ai client sono A\+S\+C\+II? Sono tutti da 1 byte? Bisogna gestire lettere accentate, ...?  
\item[\label{warning__warning000007}%
\Hypertarget{warning__warning000007}%
Globale \hyperlink{client_8h_a54b47b58f228d7bc9827d2919687e25a}{operazioni\+\_\+figlio} (char $\ast$file\+Path)]siccome l\textquotesingle{}ultima parte del messaggio e\textquotesingle{} l\textquotesingle{}unica che puo\textquotesingle{} essere piu\textquotesingle{} corta per specifica... Cosa bisogna fare in casi in cui non e\textquotesingle{} possibile garantire questo vincolo? Esempio\+: 2 caratteri possono essere divisi in\+:
\begin{DoxyItemize}
\item caratteri per parte\+: 1 1 0 0
\item oppure\+: 2 0 0 0 ~\newline
 Lo stesso problema si pone per 1, 2, 5, 6, 9, 10, ... caratteri. ~\newline
 Non posso garantire come ad esempio nel caso di 3 caratteri che sono l\textquotesingle{}ultimo numero sia inferiore\+: Esempio di suddivisione di 3 caratteri\+: 1 1 1 0. L\textquotesingle{}ultimo, come per specifica, e\textquotesingle{} l\textquotesingle{}unico di dimensione inferiore 
\end{DoxyItemize}
\item[\label{warning__warning000004}%
\Hypertarget{warning__warning000004}%
Globale \hyperlink{client_8h_a48d605ff689f470746c858648f0a98c2}{S\+I\+G\+I\+N\+T\+Signal\+Handler} (int sig)]Per ottimizzare l\textquotesingle{}uso dello H\+E\+AP nel client 0 si potrebbe prima cercare e contare quanti file sono presenti senza creare una lista concatenata e poi ricercare i file e man mano che si trovano file send\+\_\+me si puo\textquotesingle{} creare il processo figlio per inviare il file. Per fare questo B\+I\+S\+O\+G\+NA sapere se il numero di file puo\textquotesingle{} cambiare durante l\textquotesingle{}esecuzione del programma\+: se trovo 3 file e dopo un file viene cancellato cosa succede? ~\newline
 N\+O\+TA\+: questo problema puo\textquotesingle{} esserci anche nella situazione attuale...

Il client 0 deve attendere i processi figlio? La specifica indica solo che bisogna attendere il messaggio di fine dal server... Probabilmente prima bisogna attendere il messaggio di fine e poi aspettare che tutti i figlio terminino (prima di liberare la lista dei file e).

Il percorso passato al client deve essere assoluto o puo\textquotesingle{} essere relativo? Se si passa un percorso relativo chdir() fallira\textquotesingle{} alla seconda esecuzione. ~\newline
 S\+O\+L\+U\+Z\+I\+O\+NE\+: si potrebbe usare un altro chdir() a fine funzione per tornare al percorso di esecuzione iniziale anticipando il chdir() successivo.
\end{DoxyRefList}