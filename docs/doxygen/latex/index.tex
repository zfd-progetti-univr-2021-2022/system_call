Elaborato System Call per il Corso di Sistemi Operativi (2021-\/2022) 



Work In Progress.

\subsection*{Documentazione Doxygen}

E\textquotesingle{} possibile raggiungere la \href{https://zfd-progetti-univr-2021-2022.github.io/system_call/doxygen/html/index.html}{\tt documentazione generata da Doxygen cliccando qui}.

Navigando il menu\textquotesingle{} a tendina a sinistra e\textquotesingle{} possibile vedere\+:
\begin{DoxyItemize}
\item Commenti delle funzioni e dei parametri con grafici interattivi delle chiamate (\href{https://zfd-progetti-univr-2021-2022.github.io/system_call/doxygen/html/files.html}{\tt link diretto})
\item Problemi o incomprensioni delle specifiche da risolvere (\href{https://zfd-progetti-univr-2021-2022.github.io/system_call/doxygen/html/warning.html}{\tt link diretto})
\item Lista delle cose da implementare (\href{https://zfd-progetti-univr-2021-2022.github.io/system_call/doxygen/html/todo.html}{\tt link diretto})
\end{DoxyItemize}

\subsection*{Comandi utili}

Compilazione\+: 
\begin{DoxyCode}
make
\end{DoxyCode}
 \begin{quote}
Gli eseguibili finiranno nella cartella {\ttfamily dist} \end{quote}


Terminazione del client\+:
\begin{DoxyEnumerate}
\item Aprire un altro terminale
\item Eseguire il comando\+:

```bash kill -\/s S\+I\+G\+U\+S\+R1 \$(pgrep client\+\_\+0) ```

$>$ Prende l\textquotesingle{}output del comando pgrep, che restituisce il P\+ID del processo con nome client\+\_\+0, e usa il P\+ID restituito per mandargli il segnale S\+I\+G\+U\+S\+R1 di terminazione.
\end{DoxyEnumerate}

E\textquotesingle{} possibile trovare altri \href{https://zfd-progetti-univr-2021-2022.github.io/system_call/doxygen/html/md_theory_commands_commands.html}{\tt comandi utili sulla documentazione doxygen}

\subsection*{I\+PC}

Set di semafori {\ttfamily semid}\+:

\tabulinesep=1mm
\begin{longtabu} spread 0pt [c]{*{3}{|X[-1]}|}
\hline
\rowcolor{\tableheadbgcolor}\textbf{ Identificativo}&\textbf{ Valore Iniziale}&\textbf{ Descrizione  }\\\cline{1-3}
\endfirsthead
\hline
\endfoot
\hline
\rowcolor{\tableheadbgcolor}\textbf{ Identificativo}&\textbf{ Valore Iniziale}&\textbf{ Descrizione  }\\\cline{1-3}
\endhead
0 &1 &Mutex per leggere/scrivere il messaggio con il numero N di file sulla memoria condivisa \\\cline{1-3}
1 &0 &Incrementato a 2 in runtime per prevedere il ciclo successivo. Aspetta che client e server abbiano terminato operazioni sulle F\+I\+FO prima di renderle N\+ON bloccanti \\\cline{1-3}
2 &0 &Incrementato a 2 in runtime per prevedere il ciclo successivo. Aspetta che client e server abbiano reso le F\+I\+FO N\+ON bloccanti \\\cline{1-3}
3 &0 &Incrementato a 2 in runtime per prevedere il ciclo successivo. Aspetta che client e server abbiano terminato operazioni sulle F\+I\+FO prima di renderle bloccanti \\\cline{1-3}
4 &0 &Incrementato a 2 in runtime per prevedere il ciclo successivo. Aspetta che client e server abbiano reso le F\+I\+FO bloccanti \\\cline{1-3}
5 &0 &Incrementato in runtime per valere tanto quanto sono il numero di file/figli. Aspetta che tutti i processi figlio di client\+\_\+0 abbiano suddiviso il proprio file in 4 parti prima di mandarle sulle I\+PC o F\+I\+FO \\\cline{1-3}
6 &1 &Mutex per leggere/scrivere messaggi con la quarta parte di file sulla memoria condivisa \\\cline{1-3}
\end{longtabu}
