Elaborato System Call per il Corso di Sistemi Operativi (2021-\/2022) 



Work In Progress.

\subsection*{Documentazione Doxygen}

E\textquotesingle{} possibile raggiungere la \href{https://zfd-progetti-univr-2021-2022.github.io/system_call/doxygen/html/index.html}{\tt documentazione generata da Doxygen cliccando qui}.

Navigando il menu\textquotesingle{} a tendina a sinistra e\textquotesingle{} possibile vedere\+:
\begin{DoxyItemize}
\item Commenti delle funzioni e dei parametri con grafici interattivi delle chiamate (\href{https://zfd-progetti-univr-2021-2022.github.io/system_call/doxygen/html/files.html}{\tt link diretto})
\item Problemi o incomprensioni delle specifiche da risolvere (\href{https://zfd-progetti-univr-2021-2022.github.io/system_call/doxygen/html/warning.html}{\tt link diretto})
\item Lista delle cose da implementare (\href{https://zfd-progetti-univr-2021-2022.github.io/system_call/doxygen/html/todo.html}{\tt link diretto})
\end{DoxyItemize}

\subsection*{Comandi utili}

Compilazione\+: 
\begin{DoxyCode}
make
\end{DoxyCode}
 \begin{quote}
Gli eseguibili finiranno nella cartella {\ttfamily dist} \end{quote}


\begin{quote}
N\+O\+TA\+: a volte non vengono rilevate le modifiche e quindi occorre cancellare gli eseguibili dalla cartella src \end{quote}


Terminazione del client\+:
\begin{DoxyEnumerate}
\item Aprire un altro terminale
\item Eseguire il comando\+:

```bash kill -\/s S\+I\+G\+U\+S\+R1 \$(pgrep client\+\_\+0) ```

$>$ Prende l\textquotesingle{}output del comando pgrep, che restituisce il P\+ID del processo con nome client\+\_\+0, e usa il P\+ID restituito per mandargli il segnale S\+I\+G\+U\+S\+R1 di terminazione. 
\end{DoxyEnumerate}