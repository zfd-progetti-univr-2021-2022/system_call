\subsection*{Creazione della coda dei messaggi}


\begin{DoxyCode}
\textcolor{preprocessor}{#include <sys/msg.h>}
\textcolor{comment}{// Returns message queue identifier on success, or -1 error}
\textcolor{keywordtype}{int} msgget(key\_t key, \textcolor{keywordtype}{int} msgflg);
\end{DoxyCode}


Dove\+:
\begin{DoxyItemize}
\item {\ttfamily key}\+: e\textquotesingle{} una chiave I\+PC
\item {\ttfamily msgflg}\+: descrive i permessi
\begin{DoxyItemize}
\item puo\textquotesingle{} essere una delle flag del parametro {\ttfamily mode} di {\ttfamily open()}
\item {\ttfamily I\+P\+C\+\_\+\+C\+R\+E\+AT}\+: se non esiste la message queue legata alla chiave, creala
\item {\ttfamily I\+P\+C\+\_\+\+E\+X\+C\+EL}\+: usata insieme a {\ttfamily I\+P\+C\+\_\+\+C\+R\+E\+AT}, fa fallire msgget se esiste gia\textquotesingle{} una coda di messaggi legata alla chiave
\end{DoxyItemize}
\end{DoxyItemize}

Per vedere come generare la chiave andare nella \href{md_theory_generate_keys_generate_keys.html}{\tt sezione relativa qui}

Esempi\+: 
\begin{DoxyCode}
\textcolor{keywordtype}{int} \hyperlink{client_8c_ae73e6a837794db6e63f23db2d64a8130}{msqid};

ket\_t key = \textcolor{comment}{//... (generazione chiave)}

\textcolor{comment}{// A) Fai scegliere la chiave al kernel}
msqid = msgget(IPC\_PRIVATE, S\_IRUSR | S\_IWUSR);

\textcolor{comment}{// B) Ottieni identificativo della coda dei messaggi con la chiave key,}
\textcolor{comment}{//    creala se non esiste}
msqid = msgget(key, IPC\_CREAT | S\_IRUSR | S\_IWUSR);

\textcolor{comment}{// C) Ottieni identificativo della coda dei messaggi con la chiave key,}
\textcolor{comment}{//    fallisce se esiste gia'}
msqid = msgget(key, IPC\_CREAT | IPC\_EXCL | S\_IRUSR | S\_IWUSR);
\end{DoxyCode}


\subsection*{Struttura dei messaggi}

La coda dei messaggi serve per inviare e ricevere messaggi.

Un messaggio e\textquotesingle{} una struttura che possiede un campo chiamato mtype di tipo long maggiore di zero e poi i dati da inviare.

Esempio\+: 
\begin{DoxyCode}
\textcolor{keyword}{struct }mymsg \{
    \textcolor{keywordtype}{long} mtype;   \textcolor{comment}{/* tipo del messaggio */}
    \textcolor{keywordtype}{char} mtext[]; \textcolor{comment}{/* Testo */}
\};
\end{DoxyCode}


Oppure\+: 
\begin{DoxyCode}
\textcolor{keyword}{struct }mymsg \{
    \textcolor{keywordtype}{long} mtype;   \textcolor{comment}{/* tipo del messaggio */}
    \textcolor{keywordtype}{int} a;        \textcolor{comment}{/* Intero 1 */}
    \textcolor{keywordtype}{int} b;        \textcolor{comment}{/* Intero 2 */}
\};
\end{DoxyCode}


Oppure\+: 
\begin{DoxyCode}
\textcolor{keyword}{struct }mymsg \{
    \textcolor{keywordtype}{long} mtype;   \textcolor{comment}{/* tipo del messaggio */}
\};
\end{DoxyCode}


\subsection*{Inviare un messaggio}

Per scrivere un messaggio bisogna usare la system call\+: 
\begin{DoxyCode}
\textcolor{preprocessor}{#include <sys/msg.h>}
\textcolor{comment}{// Returns 0 on success, or -1 error}
\textcolor{keywordtype}{int} msgsnd(\textcolor{keywordtype}{int} msqid, \textcolor{keyword}{const} \textcolor{keywordtype}{void} *msgp, \textcolor{keywordtype}{size\_t} msgsz, \textcolor{keywordtype}{int} msgflg);
\end{DoxyCode}


Dove\+:
\begin{DoxyItemize}
\item {\ttfamily msqid}\+: identificatore della coda dei messaggi
\item {\ttfamily msgp}\+: indirizzo che punta alla struttura dei messaggi
\item {\ttfamily msgsz}\+: specifica il numero di byte contenuti nel corpo del messaggio (quindi senza contare mtype)
\item {\ttfamily msgflg}\+:
\begin{DoxyItemize}
\item 0\+: se la coda dei messaggi e\textquotesingle{} piena, questa system call si blocca in attesa di spazio libero
\item {\ttfamily I\+P\+C\+\_\+\+N\+O\+W\+A\+IT}\+: se la coda dei messaggi e\textquotesingle{} piena prosegue restituendo l\textquotesingle{}errore E\+A\+G\+A\+IN
\end{DoxyItemize}
\end{DoxyItemize}

Esempio\+: 
\begin{DoxyCode}
\textcolor{comment}{// Struttura del messaggio}
\textcolor{keyword}{struct }mymsg \{
    \textcolor{keywordtype}{long} mtype;
    \textcolor{keywordtype}{char} mtext[100]; \textcolor{comment}{/* stringa */}
\} m;

\textcolor{comment}{// messaggio di tipo 1}
m.mtype = 1;

\textcolor{comment}{// il messaggio contiene la seguente stringa}
\textcolor{keywordtype}{char} *text = \textcolor{stringliteral}{"Ciao mondo!"};

\textcolor{comment}{// copia la stringa dentro a mtext}
memcpy(m.mtext, text, strlen(text) + 1);

\textcolor{comment}{// calcolo dimensione del messaggio ignorando mtype}
\textcolor{keywordtype}{size\_t} mSize = \textcolor{keyword}{sizeof}(\textcolor{keyword}{struct }mymsg) - sizeof(long);

\textcolor{comment}{// invia il messaggio nella coda dei messaggi}
\textcolor{keywordflow}{if} (msgsnd(msqid, &m, mSize, 0) == -1)
    errExit(\textcolor{stringliteral}{"msgsnd failed"});
\end{DoxyCode}


\subsection*{Ricevere un messaggio}

La system call {\ttfamily msgrcv()} legge e rimuove il messaggio dalla queue dei messaggi\+:


\begin{DoxyCode}
\textcolor{preprocessor}{#include <sys/msg.h>}
\textcolor{comment}{// Returns number of bytes copied into msgp on success, or -1 error}
ssize\_t msgrcv(\textcolor{keywordtype}{int} msqid, \textcolor{keywordtype}{void} *msgp, \textcolor{keywordtype}{size\_t} msgsz, \textcolor{keywordtype}{long} msgtype, \textcolor{keywordtype}{int} msgflg);
\end{DoxyCode}


Dove\+:
\begin{DoxyItemize}
\item {\ttfamily msgid}\+: identificativo della coda dei messaggi
\item {\ttfamily msgp}\+: buffer dove leggere il messaggio
\item {\ttfamily msgsz}\+: dimensione del messaggio (senza {\ttfamily mtype})
\item {\ttfamily msgtype}\+: tipo del messaggio
\begin{DoxyItemize}
\item maggiore di 0\+: il primo messaggio della coda che ha il tipo uguale a msgtype viene rimosso e restituito al lettore
\item uguale a 0\+: viene letto e rimosso il primo messaggio della coda
\item minore di 0\+: viene letto e rimosso il messagio con mtype piu\textquotesingle{} piccolo e che ha valore minore o uguale al valore assoluto di {\ttfamily msgtype}
\end{DoxyItemize}
\item {\ttfamily msgflg}\+: flag
\begin{DoxyItemize}
\item 0\+: se si cerca di leggere un messaggio di tipo msgtype che non c\textquotesingle{}e\textquotesingle{} la chiamata e\textquotesingle{} bloccante.

Se la dimensione del messaggio (senza mtype) supera la dimensione definita in msgsize verra\textquotesingle{} restituito un errore.
\item {\ttfamily I\+P\+C\+\_\+\+N\+O\+W\+A\+IT}\+: se si cerca di leggere un messaggio di tipo msgtype che non c\textquotesingle{}e\textquotesingle{} la chiamata N\+ON e\textquotesingle{} bloccante. $>$ Verra\textquotesingle{} restituito l\textquotesingle{}errore E\+N\+O\+M\+SG
\item {\ttfamily M\+S\+G\+\_\+\+N\+O\+E\+R\+R\+OR}\+: Se la dimensione del messaggio (senza mtype) supera la dimensione definita in msgsize N\+ON dara\textquotesingle{} errore.

Il messaggio verra\textquotesingle{} cancellato dalla coda dei messaggi e verra\textquotesingle{} troncato per stare in msgsz bytes.
\end{DoxyItemize}
\end{DoxyItemize}

Esempio\+: 
\begin{DoxyCode}
\textcolor{comment}{// struttura dei messaggi}
\textcolor{keyword}{struct }mymsg \{
    \textcolor{keywordtype}{long} mtype;
    \textcolor{keywordtype}{char} mtext[100]; \textcolor{comment}{/* corpo del messaggio */}
\} m;

\textcolor{comment}{// Calcola la dimensione di mtext}
\textcolor{keywordtype}{size\_t} mSize = \textcolor{keyword}{sizeof}(\textcolor{keyword}{struct }mymsg) - sizeof(long);

\textcolor{comment}{// Aspetta un messaggio di tipo 1}
\textcolor{keywordflow}{if} (msgrcv(msqid, &m, mSize, 1, 0) == -1)
    errExit(\textcolor{stringliteral}{"msgrcv failed"});
\end{DoxyCode}


\subsection*{Operazioni di controllo}


\begin{DoxyCode}
\textcolor{preprocessor}{#include <sys/msg.h>}
\textcolor{comment}{// Returns 0 on success, or -1 error}
\textcolor{keywordtype}{int} msgctl(\textcolor{keywordtype}{int} msqid, \textcolor{keywordtype}{int} cmd, \textcolor{keyword}{struct} msqid\_ds *buf);
\end{DoxyCode}


Dove\+:
\begin{DoxyItemize}
\item {\ttfamily msqid}\+: identificativo della coda dei messaggi
\item {\ttfamily cmd}\+: comando da eseguire
\begin{DoxyItemize}
\item {\ttfamily I\+P\+C\+\_\+\+R\+M\+ID}\+: rimuove immediatamente la coda. Cancella tutti i messaggi e sveglia i processi in attesa con errore E\+I\+D\+RM.
\item {\ttfamily I\+P\+C\+\_\+\+S\+T\+AT}\+: salva in {\ttfamily buf} le statistiche della coda
\item {\ttfamily I\+P\+C\+\_\+\+S\+ET}\+: modifica impostazioni della coda utilizzando {\ttfamily buf} $>$ si possono modificare i campi {\ttfamily msg\+\_\+perm} e{\ttfamily msg\+\_\+qbytes}.
\end{DoxyItemize}
\item {\ttfamily buf}\+: buffer.
\end{DoxyItemize}

Esempio. Cancella la coda\+: 
\begin{DoxyCode}
\textcolor{keywordflow}{if} (msgctl(msqid, IPC\_RMID, NULL) == -1)
    errExit(\textcolor{stringliteral}{"msgctl failed"});
\textcolor{keywordflow}{else}
    printf(\textcolor{stringliteral}{"message queue removed successfully\(\backslash\)n"});
\end{DoxyCode}


Struttura msqid\+\_\+ds $\ast$buf\+: 
\begin{DoxyCode}
\textcolor{keyword}{struct }msqid\_ds \{
    \textcolor{keyword}{struct }ipc\_perm msg\_perm; \textcolor{comment}{// proprietario e permessi}
    time\_t msg\_stime; \textcolor{comment}{// tempo dell'ultimo last msgsnd()}
    time\_t msg\_rtime; \textcolor{comment}{// tempo dell'ultimo msgrcv()}
    time\_t msg\_ctime; \textcolor{comment}{// tempo dell'ultima modifica}
    \textcolor{keywordtype}{unsigned} \textcolor{keywordtype}{long} \_\_msg\_cbytes; \textcolor{comment}{// numero di byte nella coda}
    msgqnum\_t msg\_qnum; \textcolor{comment}{// numero di messaggi nella coda}
    msglen\_t msg\_qbytes; \textcolor{comment}{// numero massimo di byte inseribili nella coda}
    pid\_t msg\_lspid; \textcolor{comment}{// PID dell'ultimo msgsnd()}
    pid\_t msg\_lrpid; \textcolor{comment}{// PID dell'ultimo msgrcv()}
\};
\end{DoxyCode}
 \begin{quote}
Con {\ttfamily I\+P\+C\+\_\+\+S\+ET} si possono modificare i campi {\ttfamily msg\+\_\+perm} e{\ttfamily msg\+\_\+qbytes}. \end{quote}


Esempio. Cambiare quantita\textquotesingle{} massima di byte memorizzabili nella coda\+: 
\begin{DoxyCode}
\textcolor{keyword}{struct }msqid\_ds ds;

\textcolor{comment}{// Ottieni la struttura della coda dei messaggi}
\textcolor{keywordflow}{if} (msgctl(msqid, IPC\_STAT, &ds) == -1)
    errExit(\textcolor{stringliteral}{"msgctl"});

\textcolor{comment}{// Cambia il limite di byte massimi dell'mtext}
\textcolor{comment}{// per tutti i messaggi a 1 Kbyte}
ds.msg\_qbytes = 1024;

\textcolor{comment}{// Aggiorna la struttura nel kernel in kernel}
\textcolor{keywordflow}{if} (msgctl(msqid, IPC\_SET, &ds) == -1)
    errExit(\textcolor{stringliteral}{"msgctl"});
\end{DoxyCode}
 